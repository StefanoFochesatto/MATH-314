%%% Preamble starts here.
\documentclass{amsart}
%for the heading
\usepackage{fancyhdr, enumerate}
%for the picture. 
\usepackage{tikz}
%adjust the page width
\usepackage{amsmath}
\usepackage[margin=1in]{geometry}

\linespread{1.1}

%special commands for number sets
\def\RR{{\mathbb R}}
\def\NN{{\mathbb N}}
\def\ZZ{{\mathbb Z}}
\def\QQ{{\mathbb Q}}
\def\CC{{\mathbb C}}

\makeatletter
\newcommand{\vo}{\vec{o}\@ifnextchar{^}{\,}{}}
\makeatother

% header
\lhead{\sc  Linear Algebra: Homework 3}
\chead{\sc Stefano Fochesatto } 
\rhead{\today}
\cfoot{}
\pagestyle{fancy}

%%%% Main document starts here.

\begin{document}
\thispagestyle{fancy}


{\huge\textbf{Section 1.7:}}\\\\
%%%first problem
\noindent\textbf{Exercise 1.7.14: } In Exercises 11–14, find the value(s) of $h$ for which the vectors are linearly dependent. Justify each answer.
\begin{equation}
\begin{bmatrix} 
1\\
-1\\
3
\end{bmatrix},
\begin{bmatrix} 
-5\\
7\\
8
\end{bmatrix},
\begin{bmatrix} 
1\\
1\\
h
\end{bmatrix}
\end{equation}

 
\noindent \textbf{Solution: }By the dependence relation, to make a set of vectors linearly dependent we need to make sure that the homogeneous matrix equation has a nontrivial solution. Consider &Ax=0&\\
\begin{equation}
\begin{bmatrix} 
1&-5&1&0\\
-1&7&1&0\\
3&8&h&0
\end{bmatrix}
\end{equation}
From here we want to do row operations until we get a free variable, that is sufficient enough to show that there is a non trivial solution to the homogeneous matrix equation.\\
\begin{equation}
\begin{bmatrix} 
1&-5&1&0\\
0&2&2&0\\
3&8&h&0
\end{bmatrix}\text{ $r_{1}+r_{2}$}
\end{equation}

\begin{equation}
\begin{bmatrix} 
1&-5&1&0\\
0&1&1&0\\
3&8&h&0
\end{bmatrix}\text{ $\frac{1}{2}r_{2}$}
\end{equation}

\begin{equation}
\begin{bmatrix} 
1&0&6&0\\
0&1&1&0\\
3&8&h&0
\end{bmatrix}\text{ $-3r_{1}+r_{2}$}
\end{equation}

\begin{equation}
\begin{bmatrix} 
1&0&6&0\\
0&1&1&0\\
0&8&(h-18)&0
\end{bmatrix}\text{ $-3r_{1}+r_{3}$}
\end{equation}

\begin{equation}
\begin{bmatrix} 
1&0&6&0\\
0&1&1&0\\
0&0&(h-26)&0
\end{bmatrix}\text{ $-8r_{2}+r_{3}$}
\end{equation}
Now we can see that in order to have a non trivial solution to the homogeneous matrix equation we must have $h=26$.
\vspace{1in}







%%%second problem
\noindent\textbf{Exercise 1.7.22: } In Exercises 21 and 22, mark each statement True or False. Justify each answer on the basis of a careful reading of the text.
\begin{enumerate}[(a)]
\item Two vectors are linearly dependent if and only if they lie on a line through the origin.\\
\noindent \textbf{Answer: } True. By the dependence relation, if two vectors are linearly  dependent then one must be a multiple of the other, therefore they both lie on the same line through the origin.
\vspace{1in}

\item If a set contains fewer vectors than there are entries in the vectors then the set is linearly independent\\
\noindent \textbf{Answer: }False. Consider the set of vectors \{x,y\} such that, $x=\begin{bmatrix}1\\1\\1 \end{bmatrix}$ and $y=\begin{bmatrix}2\\2\\2 \end{bmatrix}$.  They are obviously linearly dependent, there are two vectors in the set and each vector has three entries.
\vspace{1in}

\item If $x$ and $y$ are linearly independent, and if $z$ is in $Span\{x,y\}$, then $\{x,y,z\}$ is linearly dependent.\\
\noindent \textbf{Answer: } True. We know by the definition of Span that every vector in $Span\{x,y\}$, can be written as a linear combination of $x$ and $y$.  Furthermore we know from Theorem 7 that if at-least one vector in a set is a linear combination of the others that the whole set is then linearly dependent. Thus if $z$ is in $Span\{x,y\}$, then $\{x,y,z\}$ must be linearly dependent.
\vspace{1in}

\item If a set in $\RR^{n}$ is linearly dependent, then the set contains more vectors than there are entries for each vector.\\ 
\noindent \textbf{Answer: }False. It is not entirely necessary for there to be more vectors in the set than there are entries in each vector. Consider the example in part b, there are two vectors in $\RR^{3}$ and they are linearly dependent.
\vspace{1in}





%%%third problem
\noindent\textbf{Exercise 1.7.28: } How many pivot columns must a 5x7 matrix have if its columns span $\RR^{5}$? Why?

\noindent \textbf{Solution: }We know from Theorem 4 that for the columns of the matrix to span $\RR^{5}$ there must be a pivot in every row. In order to have the columns span $\RR^{5}$ we need to have linear independence in in at least 5 dimensions, thus the pivot in each row.
\vspace{1in}




{\huge\textbf{Section 1.8:}}\\\\
%%%first problem
\noindent\textbf{Exercise 1.8.22: } In Exercises 21 and 22, mark each statement True or False. Justify each answer.\\\\
\begin{enumerate}[(a)]

\item Every matrix transformation is a linear transformation.\\
\noindent \textbf{Answer: } True. Linear transformations preserve vector addition and scalar multiplication. Every matrix  transformation does as well.\\\\
\textbf{Proof:} Suppose that $T$ is a matrix transformation such that $T(x)=A\vec{x}$ for some matrix $A$,
\begin{align}
T(c\vec{x}+d\vec{y})&=A(c\vec{x}+d\vec{y})\\
&=A(c\vec{x})+A(d\vec{y})\\
&=cA\vec{x}+dA\vec{y}\\
&=cT(\vec{x})+dT(\vec{y})
\end{align}
Since $T(c\vec{x}+d\vec{y})=cT(\vec{x})+dT(\vec{y})$ vector addition and scalar multiplication have been preserved and thus $T$ must be a linear transformation.\\
\qed
\vspace{1in}

\item The codomain of a transformation $x \to Ax$ is the set of all linear combinations of columns of $A$.\\
\noindent \textbf{Answer: } False. Suppose $Ax$ is an mxn matrix, then the codomain would be $\RR^{m}$ and the range would be set of all linear combinations of columns of $A$. There are some circumstances where the range is equal of the codomain, ie the Span of the columns of $A$ is equal to $\RR^{m}$ but that is not always true.
\vspace{1in}

\item If $T: \RR^{n} \to \RR^{m} $ is a linear transformation and if $c$ is in $\RR^{m}$, then a uniqueness question is "Is $c$ in the range of $T$?"\\
\noindent \textbf{Answer: } False. Is $c$ in the range of $T$ is an existence question not a uniqueness question. Is $T$ one-to-one, would be an example of a uniqueness question. 
\vspace{1in}

\item A linear transformation preserves the operation of vector addition and scalar multiplication.\\
\noindent \textbf{Answer: } True. By the definition found on page 65.
\vspace{1in}

\item The superposition principle is a physical description of a linear transformation.\\
\noindent \textbf{Answer: } True. The book describes the superposition principle by thinking of vectors as signals that go into a system and the output is a linear transformation of the input signal. "The system satisfies the superposition principle if whenever an input is expressed as a linear combination of such signals, the system’s response is the same linear combination of the responses to the individual signals." p(67)\\
\begin{equation}
T(c_1v_1+...+c_1v_p)=c_1T(v_1)+...+c_pT(v_p)
\end{equation} 
\vspace{1in}


\noindent\textbf{Exercise 1.8.24: } Suppose vectors $v_1,...,v_p$ span $\RR^{n}$, and let $T: \RR^{n} \to \RR^{n}$ be a linear transformation. Suppose $T(v_i)=0$ for $i=1,...,p$. Show that $T$ is the zero transformation. That is, show that if $x$ is any vector in $\RR^{n}$, then $T(x)=0$.\\

\textbf{Proof: } Suppose the set of vectors $v_1,...,v_p$ span $\RR^{n}$, $T: \RR^{n} \to \RR^{n}$ is a linear transformation, and $T(v_i)=0$ for $i=1,...,p$. Let $\vec{x} \in \RR^{n}$. Since we know that the set of vectors$v_1,...,v_p$ span $\RR^{n}$ we can write $\vec{x}$ as a linear combination of $v_1,...,v_p$\\
\begin{equation}
\vec{x}=c_1v_1+...+c_pv_p.
\end{equation}
When we try to apply the transformation $T$ to $\vec{x}$ we can see that by the superposition principle,\\
\begin{equation}
T(\vec{x})=c_1T(v_1)+...+c_pT(v_p).
\end{equation}
By substitution it is also true that,\\
\begin{equation}
T(\vec{x})=c_1(0)+...+c_p(0)
\end{equation}
\begin{equation}
T(\vec{x})=0
\end{equation}
Thus $T$ is the zero transformation.\\
\qed

 
\vspace{1in}




\noindent\textbf{Exercise 1.8.25: }Given $v\neq0$ and $p$ in $\RR^{n}$, the line through $p$ in the direction of $v$ has the parametric equation $x=p+tv$. Show that a linear transformation $T: \RR^{n} \to \RR^{n}$ maps this line onto another line or onto a single point (a degenerate line).\\\\
\noindent \textbf{Proof: } Suppose $v\neq0$ and $p$ in $\RR^{n}$, the parametric equation $x=p+tv$, and $T: \RR^{n} \to \RR^{n}$ is a linear transformation. Since $T$ is a linear transformation, when we apply it to the parametric equation $x=p+tv$ we get,\\
\begin{align}
T(x)&=T(p+tv)\\
T(x)&=T(p)+tT(v)
\end{align}
From here there are two cases.\\\\
Case 1: Suppose $T(v)=0$. Here we can see by substitution that the $T$ transformation of the line will result in a single point,
\begin{equation}
T(x)=T(p)+t(0)
\end{equation}
where $T(x)=T(p)$ for all $t$.\\\\
Case 2: Suppose $T(v)\neq0$. From here we can see that the $T$ transformation of the line will result in another line where the parametric equation is just,
\begin{equation}
T(x)&=T(p)+tT(v)
 \end{equation}
 Thus $T$ maps the line $x=p+tv$ into a line or single point.
 \\
 \qed
 
\vspace{1in}



\noindent\textbf{Exercise 1.8.29: } Define $f: \RR \to \RR$ by $f(x)=mx+b$

\begin{enumerate}[(a)]

\item Show that $f$ is a linear transformation when $b=0$\\\\
\noindent \textbf{Proof: } Suppose $f: \RR \to \RR$ is defined by $f(x)=mx+b$ and $b=0$. Consider $f(cx+dy)$ such that $x,y \in \RR$ and $c,d$ are constants.\\
\begin{align}
 f(cx+dy)&=m(cx+dy)\\
 &=mcx+mdy\\
  &=c(mx)+d(my)\\
  &=cf(x)+df(y)
 \end{align}
 Thus we have shown that when $b=0$, $f$ respects vector addition and scalar multiplication and therefore is a linear transformation.
\vspace{1in}


\item Find the property of linear transformation that is violated when $b \neq 0$\\\\
\noindent \textbf{Proof: } Suppose $f: \RR \to \RR$ is defined by $f(x)=mx+b$ and $b \neq 0$. let $c$ be any constant and $x,y \in \RR$. By substitution,
\begin{align}
 f(x)+f(y)&=(mx+b)+(my+b)\\
 &=m(x+y)+b+b\\
 &=f(x+y)+b\\
 \end{align}
 Thus when $b \neq 0$ $f$ does not respect vector addition. Consider,
 \begin{align}
 f(cx)&=m(cx)+b\\
 &=c(mx)+b\\
 &=c(mx)+\frac{c]{c}b, \text{multiplication by 1}\\
 &=c((mx)+\frac{b]{c})
 \begin{align}
 Since $(mx)+\frac{b]{c} \neq mx+b$ $f$ also dies not respect scalar multiplication.\\






\vspace{1in}


\item Why is $f$ called a linear function\\
\noindent \textbf{Answer: } $f$ called a linear function because it is a polynomial function whose degree is at most one.
\vspace{1in}






\end{document}



















