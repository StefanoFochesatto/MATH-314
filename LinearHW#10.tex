%%% Preamble starts here.
\documentclass{amsart}
%for the heading
\usepackage{fancyhdr, enumerate}
%for the picture. 
\usepackage{tikz}
%adjust the page width
\usepackage{amsmath}
\usepackage[margin=1in]{geometry}

\linespread{1.1}

%special commands for number sets
\def\RR{{\mathbb R}}
\def\NN{{\mathbb N}}
\def\ZZ{{\mathbb Z}}
\def\QQ{{\mathbb Q}}
\def\CC{{\mathbb C}}
\def\PP{{\mathbb P}}

\makeatletter
\newcommand{\vo}{\vec{o}\@ifnextchar{^}{\,}{}}
\makeatother

% header
\lhead{\sc  Linear Algebra: Homework 10}
\chead{\sc Stefano Fochesatto } 
\rhead{November 18, 2019}
\cfoot{}
\pagestyle{fancy}

%%%% Main document starts here.

\begin{document}
\thispagestyle{fancy}





{\huge\textbf{Section 5.5:}}\\
%%%first problem
\noindent\textbf{Exercise 5.5.4: } Let the matrix $A$ act on $C^2$ Find the eigenvalues and a basis for each eigenspace in $C^2$.
 \begin{equation*}
A = 
\begin{bmatrix}
5 &-2\\
1 &3
\end{bmatrix}
\end{equation*}
\noindent \textbf{Answer: }First we need to find the characteristic polynomial,
\begin{align*}
 0 &= det(A - \lambda I)\\
 &= (5-\lambda)(3 - \lambda)+2\\
 &= \lambda^2-8\lambda+17
\end{align*}
Then using the quadratic formula we get that $\lambda = 4 - i, 4 + i$. Now we want to find the eigenvector for $4 - i$. Let $\lambda = 4 - i$
\begin{align*}
0&=(A - (4 - i)I)\\
&=
\begin{bmatrix}
1 - i & -2 \\
1 & -1-i
\end{bmatrix}
\end{align*}
\begin{align*}
(1-i)x_1 - 2x_2 &= 0\\
x_1 + (-1-i)x_2 &= 0
\end{align*}
From here we can see that $x_1 = (1+i)x_2$ and $x_2$ is free. Therefore the basis for the eigenspace is,
\begin{equation*}
\{
\begin{bmatrix}
1+i\\
1
\end{bmatrix}
\}
\end{equation*}
Since we've solved for the eigenspace corresponding to the eigenvalue $\lambda = 1- i$ all we have to do to find the eigenspace for the complex conjugate $\lambda = 1+i$ is take the complex conjugate of the eigenspace we just found. Therefore the basis for the eigenspace of $\lambda = 1+i$ is,
\begin{equation*}
\{
\begin{bmatrix}
1-i\\
1
\end{bmatrix}
\}
\end{equation*}

\vspace{1in}


%%%first problem
\noindent\textbf{Exercise 5.5.8: } List the eigenvalues of $A$(using example 3). Give the angle $\delta$ of the rotation, where $-\pi < \delta \leq \pi$, and give a scale factor $r$.
\begin{equation*}
A = 
\begin{bmatrix}
\sqrt{3} &3\\
-3 &\sqrt{3}
\end{bmatrix}
\end{equation*}

\noindent \textbf{Answer: }We can see from applying example 3 that the eigenvalues for $A$ are $\lambda = \sqrt{3}+3i,\sqrt{3}-3i$ . We can find the magnitude or radius, from the following equation,
\begin{equation*}
r = |\lambda| = \sqrt{\sqrt{3}^2+(-3)^2} = 2\sqrt{3}
\end{equation*}
We can also find the angle $\delta$ using some basic trigonometry,
\begin{align*}
tan(\delta) &= \frac{-3}{\sqrt{3}}\\
\delta &= tan^{-1}(\frac{-3}{\sqrt{3}})\\
\delta &= -\frac{\pi}{3} 
\end{align*}

\vspace{1in}



%%%first problem
\noindent\textbf{Exercise 5.5.23: } 
 Let $A$ be any nxn real matrix with the property that $A^T = A$ Let $x$ be any vector in $C^n$ and let $q = \overline{x}^TAx$. the equality below show that $q$ is a real number by verifying that $\overline{q} = q$, Give a reason for each step,
\begin{equation*}
\overline{q} =  \overline{\overline{x}^TAx} = x^T\overline{Ax} = x^TA\overline{x} = (x^TA\overline{x})^{T} =\overline{x}^TA^{T}x = q  
\end{equation*}
\begin{enumerate}

\item $ \overline{\overline{x}^TAx} = x^T\overline{Ax} $\\
\noindent \textbf{Answer: }The transpose of the compliment of a vector is the same at taking the compliment of the the transposed vector, ie $\overline{x}^T = \overline{x^T}$ 
\vspace{1in}

\item $ x^T\overline{Ax} = x^TA\overline{x}$\\
\noindent \textbf{Answer: } $A$ is a real matrix, so taking the compliment, results in $\overline{A} = A$
\vspace{1in}

\item $x^TA\overline{x} = (x^TA\overline{x})^{T}$\\
\noindent \textbf{Answer: } Since $x^TA\overline{x}$ gives a scalar we can take the transpose without changing anything. 
\vspace{1in}

\item$(x^TA\overline{x})^{T} =\overline{x}^TA^{T}x $\\
\noindent \textbf{Answer: } Commutativity of transpose, $(AB)^T = B^TA^T$
\vspace{1in}

\item $\overline{x}^TA^{T}x = q $\\
\noindent \textbf{Answer: } By definition of $A$, $A^T = A$  and definition of $q$.
\vspace{1in}




{\huge\textbf{Section 6.1:}}\\
%%%first problem
\noindent\textbf{Exercise 6.1.10: } Find the unit vector in the direction of the given vector.
\begin{equation*}
v = 
\begin{bmatrix}
-6\\
4\\
3
\end{bmatrix}
\end{equation*}

\noindent \textbf{Answer: } First we need to find the magnitude of the vector,
\begin{equation*}
||v|| = \sqrt{-6^2+4^2+(-3)^2} = \sqrt{61}
\end{equation*}
Then we can get the unit vector by dividing each component by the magnitude,
\begin{equation*}
u = 
\frac{1}{\sqrt{61}}
\begin{bmatrix}
-6\\
4\\
3
\end{bmatrix}
 = 
 \begin{bmatrix}
\frac{-6}{\sqrt{61}}\\
\frac{4}{\sqrt{61}}\\
\frac{3}{\sqrt{61}}
\end{bmatrix}
\end{equation*}
\vspace{1in}


%%%first problem
\noindent\textbf{Exercise 6.1.14: } Find the distance between vectors $u$ and $z$.
\begin{equation*}
u = 
\begin{bmatrix}
0\\
-5\\
2
\end{bmatrix}
,
z = 
\begin{bmatrix}
-4\\
-1\\
8
\end{bmatrix}
\end{equation*}

\noindent \textbf{Answer: }First we subtract $z$ from $u$ to get vector $u-z$,
\begin{equation*}
u-z = 
\begin{bmatrix}
0\\
-5\\
2
\end{bmatrix}
 - 
 \begin{bmatrix}
-4\\
-1\\
8
\end{bmatrix}
= 
 \begin{bmatrix}
4\\
-4\\
-6
\end{bmatrix}
\end{equation*}

Then we take the magnitude of the resultant vector,
\begin{equation*}
||u-z|| = \sqrt{4^2+(-4)^2+(-6)^2} = 2\sqrt{17}
\end{equation*}
Therefore the distance between vectors $u$ and $z$ is $2\sqrt{17}$.
\vspace{1in}



%%%first problem
\noindent\textbf{Exercise 6.1.20: } All vectors are in $R^n$, mark each statement true or false.\\
\begin{enumerate}

\item $u\cdot v - v \cdot u = 0$\\
\noindent \textbf{Answer: } True. Since dot product is commutative, we know that $ u\cdot v  = v \cdot u$ and through a little algebra we can get the statement above. 
\vspace{1in}

\item For any scalar $c$, $||cv|| = ||cv||$\\
\noindent \textbf{Answer: } False. $||cv||$ can never be negative, while $c||v||$ can for negative values of $c$. 
\vspace{1in}

\item If $x$ is orthogonal to every vector in the subspace $W$, then $x$ is in $W^{\perp}$\\
\noindent \textbf{Answer: }True. By the definition of $W^{\perp}$.
\vspace{1in}


\item If $||u||^2 + ||v||^2 = ||u+v||^2$ then $u$ and $v$ are orthogonal.\\
\noindent \textbf{Answer: }True. By Theorem 2 in Chapter 6.
\vspace{1in}


\item For an mxm matrix $A$, vectors in the null space of $A$ are orthogonal to the vectors in the row space of $A$\\
\noindent \textbf{Answer: }True. By Theorem 3 in Chapter 6.
\vspace{1in}




%%%first problem
\noindent\textbf{Exercise 6.1.24: } Verify the parallelogram law for vectors $u$ and $v$ in $R^n$,
\begin{equation*}
||u+v||^2+||u-v||^2 = 2||u||^2+2||u||^2
\end{equation*}

\noindent \textbf{Answer: } We can see prove this through some algebra,
\begin{align*}
||u+v||^2+||u-v||^2 &=(u+v) \cdot (u+v) + (u-v) \cdot (u-v) \text{ by dot product}\\
&=(u \cdot u + v \cdot u + u \cdot v + v \cdot v) + (u \cdot u - v \cdot u - u \cdot v + v \cdot v) \\
&= 2 u\cdot u + 2 v \cdot v\\
&= 2||u||^2+2||u||^2
\end{align*}
\vspace{1in}




%%%first problem
\noindent\textbf{Exercise 6.1.27: }Suppose a vector $y$ is orthogonal to vectors $u$ snd $v$. Show that $y$ is orthogonal to the vector $u+v$
\noindent \textbf{Answer: } Since $y$ is orthogonal to both $u$ and $v$  we know that $y\cdot u = 0$ and $y \cdot v = 0$ . Now consider $y \cdot (u+v)  = y\cdot u + y\cdot v = 0$. Therefore $y$ is orthogonal to $u+v$.
\vspace{1in}



{\huge\textbf{Section 6.2:}}\\
%%%first problem
\noindent\textbf{Exercise 6.2.24: } All vectors are in $R^n$. Mark true or false. \\
\begin {enumerate}

\item Not every orthogonal set in $R^n$ is linearly independent.
\noindent \textbf{Answer: }True. An orthogonal set can contain the zero vector. 
\vspace{1in}


\item If a set $S = \{u_1, ..., u_p\}$ has the property that $u_j \cdot u_i = 0$ S.T $ i \neq j$ then $S$ is orthonormal\\
\noindent \textbf{Answer: }False. $S$ is an orthogonal set, but there is no guarantee that $S$ is orthonormal.
\vspace{1in}


\item If the columns of an m x n matrix $A$ are orthonormal, then the linear mapping $x \to Ax$ preserves lengths.\\
\noindent \textbf{Answer: }True. The columns of $A$ are formed by vectors whose magnitude is one. So $Ax$ is a fancy multiplication by one. 
\vspace{1in}


\item The orthogonal projection of $y$ onto $v$ is the same as the orthogonal projection of $y$ onto $cv$\\
\noindent \textbf{Answer:} False. Projection is dependent on the direction of the projected on vector not the magnitude, so if we let $c \leq 0$ then the projection will be zero.   
\vspace{1in}


\item An orthogonal matrix is invertible.\\
\noindent \textbf{Answer: }False. It is possible to have an orthogonal matrix that isn't square.
\vspace{1in}

\end{enumerate}


%%%first problem
\noindent\textbf{Exercise 6.2.30: } Let $U$ be an orthogonal matrix, and $V$ by interchanging some columns of $U$. Explain why $Y$ is an orthogonal matrix.\\ 
\noindent \textbf{Answer: }Since $U$ is an orthogonal matrix all pair wise columns are orthogonal to each other. If we interchange the columns this property still stands, so $Y$ must also be orthogonal. 
\vspace{1in}

 \end{document}




