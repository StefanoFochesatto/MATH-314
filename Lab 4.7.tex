%%% Preamble starts here.
\documentclass{amsart}
%for the heading
\usepackage{fancyhdr, enumerate}
%for the picture. 
\usepackage{tikz}
%adjust the page width
\usepackage{amsmath}
\usepackage[margin=1in]{geometry}

\linespread{1.1}

%special commands for number sets
\def\RR{{\mathbb R}}
\def\NN{{\mathbb N}}
\def\ZZ{{\mathbb Z}}
\def\QQ{{\mathbb Q}}
\def\CC{{\mathbb C}}
\def\PP{{\mathbb P}}

\makeatletter
\newcommand{\vo}{\vec{o}\@ifnextchar{^}{\,}{}}
\makeatother

% header
\lhead{\sc  Linear Algebra: Lab 4.7}
\chead{\sc Stefano Fochesatto } 
\rhead{\today}
\cfoot{}
\pagestyle{fancy}

%%%% Main document starts here.

\begin{document}
\thispagestyle{fancy}





{\huge\textbf{Lab 4.7}}\\\\
%%%first problem

\noindent\textbf{Exercise 1: }Let $B = \{b_1,b_2\}$ and $C = \{c_1,c_2\}$ be bases for $\RR^2$. find the change of coordinates matrix from $B \to C$ and from $C \to B$
\begin{equation*}
b_1 =
\begin{bmatrix}
1\\
4
\end{bmatrix},
b_2 =
\begin{bmatrix}
-2\\
3
\end{bmatrix},
c_1 =
\begin{bmatrix}
-2\\
3
\end{bmatrix},
c_2 =
\begin{bmatrix}
4\\
4
\end{bmatrix},
\end{equation*}

\begin{enumerate}
\item Set up the matrix $M$ whose columns are $[c_1,c_2,b_1.b_2]$\\
\noindent \textbf{Answer: } Consider the following matrix,
\begin{equation*}
M =
\begin{bmatrix}
-2&4&1&-2\\
3&4&4&3
\end{bmatrix},
\end{equation*}
\vspace{.1in}

\item Row reduce matrix $M$.\\
\noindent \textbf{Answer: } Row reducing matrix M,
\begin{align*}
\begin{bmatrix}
-2&4&1&-2\\
3&4&4&3
\end{bmatrix}
&\approx
\begin{bmatrix}
3&4&4&3\\
0&\frac{20}{3}&\frac{11}{3}&0
\end{bmatrix} \text{ Row swap and row replacement at the same time.}\\
&\approx
\begin{bmatrix}
3&4&4&3\\
0&1&\frac{11}{20}&0
\end{bmatrix}\\
&\approx
\begin{bmatrix}3&0&\frac{9}{5}&3\\ 0&1&\frac{11}{20}&0\end{bmatrix}\\
&\approx
\begin{bmatrix}1&0&\frac{3}{5}&1\\ 0&1&\frac{11}{20}&0\end{bmatrix}
\end{align*}
\vspace{.1in}


\item Calculate ${P_c}^{-1}$, where  ${P_c} = [c_1,c_2]$.\\
\noindent \textbf{Answer: }Consider,
\begin{equation*}
{P_c} = 
\begin{bmatrix}
-2&4\\
3&4
\end{bmatrix}
\end{equation*}
We know that probably the fastest way to calculate the inverse of a 2x2 matrix is to just multiply the whole thing by the reciprocal of the determinant. So,
\begin{align*}
{P_c}^{-1} &= \frac{1}{-20}
\begin{bmatrix}
-2&4\\
3&4
\end{bmatrix}\\
&= \begin{bmatrix}-\frac{1}{5}&\frac{1}{5}\\ \frac{3}{20}&\frac{1}{10}\end{bmatrix}
\end{align*}

\vspace{.1in}

\item Calculate $N = {P_c}^{-1}P_b$.\\
\noindent \textbf{Answer: } Note that $P_b = [b_1,b_2]$ now through some matrix algebra we see that,
\begin{align*}
N &= \begin{bmatrix}-\frac{1}{5}&\frac{1}{5}\\ \frac{3}{20}&\frac{1}{10}\end{bmatrix}
*\begin{bmatrix}
1&-2\\
4&3
\end{bmatrix}\\
&=
\begin{bmatrix}\left(-\frac{1}{5}\right)\cdot \:1+\frac{1}{5}\cdot \:4&\left(-\frac{1}{5}\right)\left(-2\right)+\frac{1}{5}\cdot \:3\\ \frac{3}{20}\cdot \:1+\frac{1}{10}\cdot \:4&\frac{3}{20}\left(-2\right)+\frac{1}{10}\cdot \:3\end{bmatrix}\\
&=
\begin{bmatrix}\frac{3}{5}&1\\ \frac{11}{20}&0\end{bmatrix}
\end{align*}
\vspace{.1in}

\item What do you notice about the last two columns of the reduced echelon form of $M$ and the matrix $N$\\
\noindent \textbf{Answer: }They are the same, this makes sense because when we row reduce matrix $M$ we are finding the change of coordinates matrix from $C\to B$ and it should be clear that when when we multiply $P_b$ by the ${P_c}^{-1}$ we are essentially doing the same thing, for example take a vector written in terms of Basis $C$ when we multiply by ${P_c}^{-1}$ we are converting it to the standard basis, and finally we multiply it by $P_b$ to get it in terms of Basis $B$.
\vspace{.1in}
\end{enumerate}













%%This is ##2
\noindent\textbf{Exercise 2: }Let $B = \{b_1,b_2\}$ and $C = \{c_1,c_2\}$ be bases for $\RR^2$. find the change of coordinates matrix from $B \to C$ and from $C \to B$
\begin{equation*}
b_1 =
\begin{bmatrix}
9\\
9
\end{bmatrix},
b_2 =
\begin{bmatrix}
-12\\
6
\end{bmatrix},
c_1 =
\begin{bmatrix}
-5\\
13
\end{bmatrix},
c_2 =
\begin{bmatrix}
14\\
2
\end{bmatrix},
\end{equation*}

\begin{enumerate}
\item Set up the matrix $M$ whose columns are $[c_1,c_2,b_1.b_2]$\\
\noindent \textbf{Answer: }
\begin{align*}
M &=
\begin{bmatrix}
-5&14&9&-12\\
13&2&9&6
\end{bmatrix}\\
&\approx
\begin{bmatrix}1&0&\frac{9}{16}&\frac{9}{16}\\ 0&1&\frac{27}{32}&-\frac{21}{32}\end{bmatrix}
\end{align*}
\vspace{.1in}

\item Calculate ${P_c}^{-1}$, where  ${P_c} = [c_1,c_2].$\\
\noindent \textbf{Answer: } Note that,
\begin{equation*}
{P_c} = 
\begin{bmatrix}
-5&14\\
13&2
\end{bmatrix}
\end{equation*}
Same as before, since we have a 2x2 matrix we can just multiply by the reciprocal of the determinant.
\begin{align*}
{P_c}^{-1} &= \frac{1}{-192} 
*
\begin{bmatrix}
-5&14\\
13&2
\end{bmatrix}\\
&=
\begin{bmatrix}-\frac{1}{96}&\frac{7}{96}\\ \frac{13}{192}&\frac{5}{192}\end{bmatrix}
\end{align*}
\vspace{.1in}

\Item Calculate $N = {P_c}^{-1}P_b$.\\
\noindent \textbf{Answer: } Note that $P_b = [b_1,b_2]$ now through some matrix algebra we see that,
\begin{align*}
N &=
\begin{bmatrix}-\frac{1}{96}&\frac{7}{96}\\ \frac{13}{192}&\frac{5}{192}\end{bmatrix}
*
\begin{bmatrix}
9&-12\\
9&6
\end{bmatrix}\\
&=
\begin{bmatrix}\left(-\frac{1}{96}\right)\cdot \:9+\frac{7}{96}\cdot \:9&\left(-\frac{1}{96}\right)\left(-12\right)+\frac{7}{96}\cdot \:6\\ \frac{13}{192}\cdot \:9+\frac{5}{192}\cdot \:9&\frac{13}{192}\left(-12\right)+\frac{5}{192}\cdot \:6\end{bmatrix}\\
&=
\begin{bmatrix}\frac{9}{16}&\frac{9}{16}\\ \frac{27}{32}&-\frac{21}{32}\end{bmatrix}
\end{align*}
\vspace{.1in}
\end{enumerate}



\noindent\textbf{Exercise 3: } What is the pattern that we have established in exercises 1 and 2?\\
\noindent \textbf{Answer: } I believe I have already addressed this question in exercise 1.5, and quite thoroughly as well.
\vspace{.5in}

\noindent\textbf{Exercise 4: }Suppose we are given bases $B = \{b_1,b_2,...b_n\}$ and $C = \{c_1,c_2,...,c_n\}$ for a finite dimensional vector space $V = R_n.$ Consider now the reduced row echelon form $M′$ of the matrix $M = [c_1,c_2,...,c_n,b_1,b_2,...b_n].$ Denote the columns of $M′$ by $m_1$. Thus for $i = 1,2,...,n$, $m_i = e_i$. What then are the vectors $m_i$ for $i = n+1,n+2,...,2n?$\\
\noindent \textbf{Answer: } The vectors  $m_i$ for $i = n+1,n+2,...,2n$ form the change of coordinate matrix from $C \to B$ In other words they are the $C$- coordinate vectors in the basis $B$.
\vspace{.1in}

\noindent\textbf{Exercise 5: } For each Matrix $M$ obtained in exercises 1 and 2 compare ${P_c}^{-1}M$ and ${P_c}^{-1}P_b$\\
\noindent \textbf{Answer: }Consider the following matrices from exercise 1,
\begin{equation*}
{P_c}^{-1}M=\begin{bmatrix}-\frac{1}{5}&\frac{1}{5}\\ \frac{3}{20}&\frac{1}{10}\end{bmatrix}*\begin{bmatrix}1&0&\frac{3}{5}&1\\ 0&1&\frac{11}{20}&0\end{bmatrix}
=\begin{bmatrix}-\frac{1}{5}&\frac{1}{5}&-\frac{1}{100}&-\frac{1}{5}\\ \frac{3}{20}&\frac{1}{10}&\frac{29}{200}&\frac{3}{20}\end{bmatrix}
\end{equation*}
\begin{equation*}
{P_c}^{-1}P_b =\begin{bmatrix}-\frac{1}{5}&\frac{1}{5}\\ \frac{3}{20}&\frac{1}{10}\end{bmatrix}*\begin{bmatrix}
1&-2\\
4&3
\end{bmatrix}
=\begin{bmatrix}\frac{3}{5}&1\\ \frac{11}{20}&0\end{bmatrix}
\end{equation*}
Consider the following matrices from exercise 2,
\begin{equation*}
{P_c}^{-1}M=\begin{bmatrix}-\frac{1}{96}&\frac{7}{96}\\ \frac{13}{192}&\frac{5}{192}\end{bmatrix}*\begin{bmatrix}1&0&\frac{9}{16}&\frac{9}{16}\\ 0&1&\frac{27}{32}&-\frac{21}{32}\end{bmatrix}
= \begin{bmatrix}-\frac{1}{96}&\frac{7}{96}&\frac{57}{1024}&-\frac{55}{1024}\\ \frac{13}{192}&\frac{5}{192}&\frac{123}{2048}&\frac{43}{2048}\end{bmatrix}


\end{equation*}
\begin{equation*}
{P_c}^{-1}P_b =\begin{bmatrix}-\frac{1}{96}&\frac{7}{96}\\ \frac{13}{192}&\frac{5}{192}\end{bmatrix}*\begin{bmatrix}
9&-12\\
9&6
\end{bmatrix}
=\begin{bmatrix}\frac{9}{16}&\frac{9}{16}\\ \frac{27}{32}&-\frac{21}{32}\end{bmatrix}
\end{equation*}
They both essentially represent the same operation but because of the notation $M$ doesn't give us a useful answer when we change coordinates.\\
\vspace{.1in}


\noindent\textbf{Exercise 6: } What us the result of ${P_c}^{-1}P_b[x_b]$\\
\noindent \textbf{Answer: } Well we can see that the following statement simplifies,
\begin{equation*}
{P_c}^{-1}P_b[x_b]={P_c}^{-1}x=[x]_c
\end{equation*}
and this is due to the fact that any $P_z$ change-of-coordinates matrix will convert $z$-coordinate vectors, ie $[x_z]$ into standard ones. It goes the opposite way for the inverse of a change-of-coordinates matrix, goes from standard to $z$-coordinate vectors.
\vspace{.1in}

\noindent\textbf{Exercise 7: } Explain why the change-of-coordinates matrix from $C \to B$ is equal to the inverse of the  change-of-coordinates matrix from $B \to C$\\
\noindent \textbf{Answer: } This is discussed in the previous question. except instead of going in between the standard basis and the a $z$ basis its now a $c$ basis and a $b$ basis.
\end{document}



















