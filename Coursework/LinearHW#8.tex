%%% Preamble starts here.
\documentclass{amsart}
%for the heading
\usepackage{fancyhdr, enumerate}
%for the picture. 
\usepackage{tikz}
%adjust the page width
\usepackage{amsmath}
\usepackage[margin=1in]{geometry}

\linespread{1.1}

%special commands for number sets
\def\RR{{\mathbb R}}
\def\NN{{\mathbb N}}
\def\ZZ{{\mathbb Z}}
\def\QQ{{\mathbb Q}}
\def\CC{{\mathbb C}}
\def\PP{{\mathbb P}}

\makeatletter
\newcommand{\vo}{\vec{o}\@ifnextchar{^}{\,}{}}
\makeatother

% header
\lhead{\sc  Linear Algebra: Homework 8}
\chead{\sc Stefano Fochesatto } 
\rhead{\today}
\cfoot{}
\pagestyle{fancy}

%%%% Main document starts here.

\begin{document}
\thispagestyle{fancy}





{\huge\textbf{Section 4.5:}}\\\\
%%%first problem
\noindent\textbf{Exercise 4.5.19: }
\begin{enumerate}

\item The number of pivot columns of a matrix equals the dimension of it's column space.\\
\noindent \textbf{Answer: } True. By Theorem 6 which says that the pivot columns of a matrix $A$ for a basis for the $Col(A)$. Therefore it must be true that the number of pivot columns is equal to the number of pivots.
\vspace{1in}


\item A plane in $\RR^{3}$ is a two-dimensional subspace pf $\RR^3$.\\
\noindent \textbf{Answer: } False. Consider a plane that does not go through the origin, It cannot be a subspace.
\vspace{1in}


\item The dimension of vector space $\PP_4$ is 4.\\
\noindent \textbf{Answer: } False. $\PP_4$ is all fourth degree polynomials, which have 5 coefficients. Therefore the dimension of $\PP_4$ is 5.\\
\vspace{1in}


\item If the $Dim V = n$ and $S$ is linearly independent set in $V$ , then $S$ is a basis for $V$.\\
\noindent \textbf{Answer: } False. It is possible that the set $S$ does not span $V$ therefore it is not always a basis. Consider any $V= \RR^2$ and $S = [1,1]$, S is a linearly independent set that contains one vector, but It doesn't span $\RR^2$  therefore it is not a basis.
\vspace{1in}


\item If a set $\{v_1,...v_p \}$ spans a finite-dimensional vector space $V$ and if $T$ is a set of more than $p$ vectors in $V$, then $T$ is linearly dependent.\\
\noindent \textbf{Answer: } True. By spanning set theorem.
\vspace{1in}
 
 
 \end{enumerate}

%%%second problem
\noindent\textbf{Exercise 4.5.22: } The first four Laguerre polynomials are $1, 1-t,2-4t+t^2$, and $6-18t+9t^2-t^3$ show that these polynomials for a basis for $\PP_3$\\
\noindent \textbf{Solution: } Consider the Laguerre polynomials in the form of coordinate vectors,\\
\begin{equation}
\begin{bmatrix}
1 &1&2&6\\
0&-1&-4&-18\\
0&0&1&9\\
0&0&0&-1
\end{bmatrix}
\end{equation}
It is clear that we have 4 linearly independent vectors, which means that the given vector span $\PP_3$. Through the spanning set theorem we can say the form a basis for. $\PP_3$ because the set is linearly independent. 
\vspace{1in}

%%%third problem
\noindent\textbf{Exercise 4.5.24: }Let $B$ be a basis for $\PP_2$ consisting of the first three Laguerre polynomials, and let $p(t) = 7-8t+3t^2$. Find the coordinate vector of $p$ relative to $B$. \\

\noindent \textbf{Solution: } Consider the following matrix equation,
\begin{equation}
\begin{bmatrix}
1 &1&2\\
0&-1&-4\\
0&0&1
\end{bmatrix}
[P]_b = 
\begin{bmatrix}
7\\
-8\\
3
\end{bmatrix}
\end{equation}
All we have to do now is solve the system. Consider the following matrix,
\begin{equation}
\begin{bmatrix}
1 &0&0&5\\
0&1&0&-4\\
0&0&1&3
\end{bmatrix}
\end{equation}
So we can see that the solution i.e, $[P]_b = \begin{bmatrix}
5\\
-4\\
3
\end{bmatrix}$ 
\vspace{1in}




{\huge\textbf{Section 4.6:}}\\\\
%%%first problem
\noindent\textbf{Exercise 4.6.17: } A is an n x m matrix.\\
\begin{enumerate}

\item The row space of $A$ is the same as the column space of $A^T$.\\
\noindent \textbf{Answer: } True. The rows of $A$ are the columns of $A^T$, and pivot positions stay the same through transpose.
\vspace{1in}

\item If $B$ is any echelon form of $A$, and if $B$ has three nonzero rows, then the first three rows of $A$ form a basis for Row $A$.\\
\noindent \textbf{Answer: } False. The non zero rows of $B$ form the basis for Row $A$.
\vspace{1in}

\item The dimensions of the row space and the columns space of $A$ are the same, even if $A$ is not a square.\\
\noindent \textbf{Answer: } True. Pivot positions determine the dimensions of the column and row spaces. Since a pivot column is also a pivot row the have the same dimension.
\vspace{1in}

\item The sum fo the dimensions of the row space and the null space $A$ equals the number of rows in $A$.\\
\noindent \textbf{Answer: } False. By the rank-nullity that the dimension of the row space is equal to $n - dim Nul A$. Let $n = 5$ and $dim Nul A = 2$  then $dim row A  = 3$ and $3 \neq 2$.
\vspace{1in}

\item On a computer, row operation can change the apparent rank of a matrix.\\
\noindent \textbf{Answer: } True.  Consider the numerical note in page 238.
\vspace{1in}

\end{enumerate}

%%%second problem
\noindent\textbf{Exercise 4.6.20: }Suppose a non homogeneous system of six linear equations in eight unknown has a solution, with two free variables. Is it possible to change some constants on the equations right side to make the new system inconsistent. 
\noindent \textbf{Solution: } First let's describe the dimensions of all the spaces of the associated matrix. Let $A$ be the associated matrix, since there are two free variable we know that the $dim null A  = 2$. Then we can use the rank-nullity theorem to find the rank of the matrix, $rank A = 8 - 2 = 6$ Thus we know that the $dim Col A  = 6$ which means there are pivot positions. Since there is a pivot in every row ,ie nor zero rows we cannot make the system inconsistent by changing the RHS.
\vspace{1in}


%%%third problem
\noindent\textbf{Exercise 4.6.28: } Justify the following equalities(let A be m x n).
\begin{equation}
dim Row A + dim Nul A  = n
\end{equation}
\begin{equation}
dim Col A + dim Nul A^T  = M
\end{equation}
\noindent \textbf{Solution: } We can explain the first equality by simply noting that the $dim Row A = dim Col A$ this comes from the property that the dimension for both spaces is determined by the number of pivot position (a pivot is a pivot for a row and a column just the same.) The the equality become the same as the rank - nullity theorem.\\

The second is just a variation once note that $dim Col A = dim Row A^T$.
\vspace{1in}





{\huge\textbf{Section 4.7:}}\\\\
%%%first problem

\noindent\textbf{Exercise 4.7.7: } Let $B = \{b_1,b_2\}$ and $C = \{c_1,c_2\}$ be bases for $\RR^2$. find the change of coordinates matrix from $B \to C$ and from $C \to B$
\begin{equation*}
b_1 =
\begin{bmatrix}
7\\
5
\end{bmatrix},
b_2 =
\begin{bmatrix}
-3\\
-1
\end{bmatrix},
c_1 =
\begin{bmatrix}
1\\
-5
\end{bmatrix},
c_2 =
\begin{bmatrix}
-2\\
2
\end{bmatrix},
\end{equation*}

\noindent \textbf{Solution: } In order to solve for the change of coordinates matrix we take the same approach we used for finding the inverse of a matrix, for example consider, the change of coordinates matrix from $B \to C$
 \begin{align}
 \begin{bmatrix}
 1&-2\\
 -5&2
 \end{bmatrix}
&\approx 
  \begin{bmatrix}
 7&-3\\
 5&-1
 \end{bmatrix}\\
  \begin{bmatrix}
 1&-2\\
 0&-8
 \end{bmatrix}
&\approx 
  \begin{bmatrix}
 7&-3\\
 40&-16
 \end{bmatrix}\\
   \begin{bmatrix}
 1&-2\\
 0&1
 \end{bmatrix}
&\approx 
  \begin{bmatrix}
 7&-3\\
 -5&2
 \end{bmatrix}\\
   \begin{bmatrix}
 1&0\\
 0&1
 \end{bmatrix}
&\approx 
  \begin{bmatrix}
-3&1\\
 -5&2
 \end{bmatrix}
 \end{align}
 Thus the change of coordinates matrix from $B \to C$ is,
\begin{equation*}
  \begin{bmatrix}
-3&1\\
 -5&2
 \end{bmatrix}
\end{equation*}.
The same technique is done to get the change of coordinates matrix from $C \to B$,
\begin{align}
\begin{bmatrix}
 7&-3\\
 5&-1
 \end{bmatrix}
&\approx 
 \begin{bmatrix}
 1&-2\\
 -5&2
 \end{bmatrix}\\
\begin{bmatrix}
 1&0\\
0&1
 \end{bmatrix}
&\approx 
 \begin{bmatrix}
 -2&1\\
 -5&3
 \end{bmatrix}
 \end{align}
Thus the change of coordinates matrix from $C \to B$,
\begin{equation*}
 \begin{bmatrix}
 -2&1\\
 -5&3
 \end{bmatrix}
\end{equation*}
\vspace{1in}




%%%second problem
\noindent\textbf{Exercise 4.7.9: }Let $B = \{b_1,b_2\}$ and $C = \{c_1,c_2\}$ be bases for $\RR^2$. find the change of coordinates matrix from $B \to C$ and from $C \to B$
\begin{equation*}
b_1 =
\begin{bmatrix}
-6\\
-1
\end{bmatrix},
b_2 =
\begin{bmatrix}
2\\
0
\end{bmatrix},
c_1 =
\begin{bmatrix}
2\\
-1
\end{bmatrix},
c_2 =
\begin{bmatrix}
6\\
-2
\end{bmatrix},
\end{equation*}

\noindent \textbf{Solution: } Here we can take the same approach as we did for the last problem. Consider the change of coordinates matrix from $B \to C$,
\begin{align}
 \begin{bmatrix}
 2&6\\
 -1&-2
 \end{bmatrix}
&\approx 
  \begin{bmatrix}
 -6&2\\
 -1&0
 \end{bmatrix}\\
 \begin{bmatrix}
 1&3\\
 -1&-2
 \end{bmatrix}
&\approx 
  \begin{bmatrix}
 -3&1\\
 -1&0
 \end{bmatrix}\\
 \begin{bmatrix}
 1&3\\
 0&1
 \end{bmatrix}
&\approx 
  \begin{bmatrix}
 -3&1\\
 -4&1
 \end{bmatrix}\\
 \begin{bmatrix}
 1&0\\
 0&1
 \end{bmatrix}
&\approx 
  \begin{bmatrix}
 9&-2\\
 -4&1
 \end{bmatrix}
 \end{align}
Thus the  change of coordinates matrix from $B \to C$ is,
\begin{equation*}
  \begin{bmatrix}
 9&-2\\
 -4&1
 \end{bmatrix}
\end{equation*}
Consider the change of coordinates matrix from $C \to B$,

\begin{align}
 \begin{bmatrix}
 -6&2\\
 -1&0
 \end{bmatrix}
&\approx 
 \begin{bmatrix}
 2&6\\
 -1&-2
 \end{bmatrix}\\
  \begin{bmatrix}
 1&0\\
 0&1
 \end{bmatrix}
&\approx 
 \begin{bmatrix}
 1&2\\
 4&9
 \end{bmatrix}
 \end{align}
 Thus the change of coordinates matrix from $B \to C$,
 \begin{equation*}
 \begin{bmatrix}
 1&2\\
 4&9
 \end{bmatrix}
\end{equation*}

\vspace{1in}


%%%third problem
\noindent\textbf{Exercise 4.7.14: } In $P_2$ find the change of coordinates matrix from the basis $B = \{1-3t^2,2+t-5t^2,1+2t\}$ to the standard basis $C = \{1,t,t^2\}$ then find the $B$-coordinate vector for $-1+2t$.
\noindent \textbf{Solution: } Since $C$ is the standard basis, the coordinates of a polynomial from the standard basis are simply the coefficients. Therefore,
\begin{equation*}
[b_1]1 = 
\begin{bmatrix}
1\\
0\\
-3
\end{bmatrix}
\end{equation*}

\begin{equation*}
[b_2]_c = 
\begin{bmatrix}
2\\
1\\
-5
\end{bmatrix}
\end{equation*}
\begin{equation*}
[b_3]_c = 
\begin{bmatrix}
1\\
2\\
0
\end{bmatrix}
\end{equation*}
Therefore the change of coordinates matrix from $B\to C$ is,
\begin{equation*}
\begin{bmatrix}
1&2&1\\
0&1&2\\
-3&-5&0
\end{bmatrix}
\end{equation*}
In order to calculate $B$-coordinate vector for $-1+2t$ we need the change of coordinates matrix from $C\to B$. A quick way to do so in this case is to just take the inverse of the change of coordinates matrix from $B\to C$.Therefore the change of coordinates matrix from $C\to B$,
\begin{equation*}
\begin{bmatrix}
10&-5&3\\
-6&3&-2\\
3&-1&1
\end{bmatrix}
\end{equation*}
Then all we need to do to calculate the  $B$-coordinate vector for $-1+2t$ is just multiply the coordinates to $-1+2t$ by the change of coordinate matrix from $C\to B$,
\begin{equation*}
\begin{bmatrix}
10&-5&3\\
-6&3&-2\\
3&-1&1
\end{bmatrix}
\begin{bmatrix}
-1\\
2\\
0
\end{bmatrix}
 = 
 \begin{bmatrix}
-20\\
12\\
-5
\end{bmatrix}
\end{equation*}
\vspace{1in}


\end{document}



















