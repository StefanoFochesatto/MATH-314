%%% Preamble starts here.
\documentclass{amsart}
%for the heading
\usepackage{fancyhdr, enumerate}
%for the picture. 
\usepackage{tikz}
%adjust the page width
\usepackage{amsmath}
\usepackage[margin=1in]{geometry}

\linespread{1.1}

%special commands for number sets
\def\RR{{\mathbb R}}
\def\NN{{\mathbb N}}
\def\ZZ{{\mathbb Z}}
\def\QQ{{\mathbb Q}}
\def\CC{{\mathbb C}}

% header
\lhead{\sc  Linear Algebra: Homework 2}
\chead{\sc Stefano Fochesatto } 
\rhead{\today}
\cfoot{}
\pagestyle{fancy}

%%%% Main document starts here.

\begin{document}
\thispagestyle{fancy}


{\huge\textbf{Section 1.4:}}\\\\
%%%first problem
\noindent\textbf{Exercise 1.4.12: } Given $A$ and $b$ in Exercises 11 and 12, write the augmented matrix for the linear system that corresponds to the matrix equation $Ax=b$. Then solve the system and write the solution as a vector.
\begin{equation}
A=
\begin{bmatrix}
1&2&1\\
-3&-1&2\\
0&5&3
\end{bmatrix}\\
, b=
\begin{bmatrix}
0\\
1\\
-1
\end{bmatrix}
\end{equation}

\noindent \textbf{Solution: }
The corresponding augmented matrix looks like this,

\begin{equation}
\begin{bmatrix}
1&2&1&0\\
-3&-1&2&1\\
0&5&3&-1
\end{bmatrix}
\end{equation}
Then we want to get the matrix in RREF to find the solution,

\begin{equation}
\begin{bmatrix}
1&2&1&0\\
0&5&5&1\\
0&5&3&-1
\end{bmatrix},  3r_1+r_2
\end{equation}

\begin{equation}
\begin{bmatrix}
1&2&1&0\\
0&1&1&\frac{1}{5}\\
0&5&3&-1
\end{bmatrix},  \frac{1}{5}r_2
\end{equation}

\begin{equation}
\begin{bmatrix}
1&2&1&0\\
0&1&1&\frac{1}{5}\\
0&0&-2&-2
\end{bmatrix}, -5r_2+r_3
\end{equation}

\begin{equation}
\begin{bmatrix}
1&0&-1&-\frac{2}{5}\\
0&1&1&\frac{1}{5}\\
0&0&-2&-2
\end{bmatrix}, -2r_2+r_1
\end{equation}

\begin{equation}
\begin{bmatrix}
1&0&-1&-\frac{2}{5}\\
0&1&1&\frac{1}{5}\\
0&0&1&1
\end{bmatrix}, -\frac{1}{2}r_3
\end{equation}

\begin{equation}
\begin{bmatrix}
1&0&0&\frac{3}{5}\\
0&1&0&-\frac{4}{5}\\
0&0&1&1
\end{bmatrix}, r_3+r_1, -r_3+r_2
\end{equation}
Now that we have achieved RREF it is clear to see that,

\begin{align}
x&=
\begin{bmatrix}
\frac{3}{5}\\
-\frac{4}{5}\\
1
\end{bmatrix}
\end{align}






\vspace{1in}






%%%second problem
\noindent\textbf{Exercise 1.4.24: } In Exercises 23 and 24, mark each statement True or False. Justify
each answer.
\begin{enumerate}[(a)]
\item Every matrix equation $Ax=b$ corresponds to a vector equation with the same solution set.\\

\noindent \textbf{Answer: }
True. The matrix equation, vector equation and augmented matrix forms are all equivalent forms to the same problem.

\vspace{1in}

\item Any linear combination of vectors can always be written in the form $Ax$ for a suitable matrix $A$ and vector $x$. \\

\noindent \textbf{Answer: }
True. From our definition of the matrix equation from p.35 that $Ax$ is the linear combination of the columns of $A$ using the corresponding entries in $x$ as weight. 
\vspace{1in}

\item The solution set of a linear system whose augmented matrix is $\begin{bmatrix} a_1&a_2&a_3&b \end{bmatrix}$ is the same as the solution set of 􏰤$Ax=b$ if $A=\begin{bmatrix} a_1&a_2&a_3 \end{bmatrix}$ . \\

\noindent \textbf{Answer: }
True. See Theorem 3 on p.36.
\vspace{1in}

\item If the equation $Ax=b$ is inconsistent, then $b$ is not in the set spanned by the columns of $A$. \\

\noindent \textbf{Answer: }
True. If the equation $Ax=b$ is inconsistent, then the set spanned by the columns of $A$ is empty.

\vspace{1in}


\item If the augmented matrix 􏰤\begin{bmatrix} A&b \end{bmatrix} has a pivot position in every row, then the equation $Ax=b$ is inconsistent. \\

\noindent \textbf{Answer: }
False. If the augmented matrix 􏰤\begin{bmatrix} A&b \end{bmatrix} has a pivot position in every row, we cannot say whether the system is inconsistent or consistent. If the statement were to be altered stating that the coefficient matrix 􏰤\begin{bmatrix} A\end{bmatrix} has a pivot position in every row then we could conclude that the system is consistent.
\vspace{1in}


\end{enumerate}









%%%third problem
\noindent\textbf{Exercise 1.4.30: }Construct a 3x3 matrix, not in echelon form, whose columns do not span $\RR^{3}$. 
Show that the matrix you construct has the desired property.\\\\

\noindent \textbf{Solution: }
To construct a matrix with whose columns do not span $\RR^{3}$ all we have to do is make sure the last row reduces to all zeroes, that way the vector 
\begin{bmatrix}
0\\
0\\
1
\end{bmatrix} 
is excluded from the span. Let $A$ be,
\begin{equation}
A=
\begin{bmatrix}
0&1&0\\
1&0&1\\
1&1&1
\end{bmatrix},
\end{equation}
Then we get $A$ in RREF,

\begin{equation}
\begin{bmatrix}
1&0&1\\
0&1&0\\
0&0&0
\end{bmatrix},r_1-r_3,  r_2-r_3, r_1\to r_2
\end{equation}

From here it is clear that 
\begin{bmatrix}
0\\
0\\
1
\end{bmatrix} 
is not in the span of the three columns of the matrix $A$.
\vspace{1in}





{\huge\textbf{Section 1.5:}}\\\\
%%%first problem
\noindent\textbf{Exercise 1.5.24: }In Exercises 23 and 24, mark each statement True or False. Justify each answer.

\begin{enumerate}[(a)]
\item If $x$ is a nontrivial solution of $Ax=0$ then every entry in $x$ is nonzero \\
\noindent \textbf{Answer: }False. Just one entry needs to not be zero in order for the solution to be considered nontrivial.
\vspace{1in}

\item The equation $x= x_2u+x_3v,$ with $x_2$ and $x_3$ free (and neither $u$ nor $v$ a multiple of each other), describe a plane through the origin. \\
\noindent \textbf{Answer: }True. The solution set is defined as Span\{u,v\} , and we can see this in section 1.3 pg. 30.
\vspace{1in}

\item The equation $Ax=b$ is homogenous if the zero vector is a solution\\
\noindent \textbf{Answer: }True. By the definition of homogenous p.43. If zero vector is in solution we can say that $A0=b$ which then implies that 
\vspace{1in}

\item The effect of adding $p$ to a vector is to move the vector in a direction parallel to $p$.\\
\noindent \textbf{Answer: }False. The effect of adding $p$, or a particular solution to the general solution gives a solution set that is parallel to the general solution set, and not the particular solution.

\vspace{1in}


\end{enumerate}
\vspace{1in}







%%%second problem
\noindent\textbf{Exercise 1.5.26: } Suppose $Ax=b$ has a solution. Explain why the solution is unique precisely when $Ax=0$ has only the trivial solution.\\\\
\noindent \textbf{Solution: } The theorem in the book states that "The homogeneous equation $Ax=0$ has a nontrivial solution if and only if  the equation has at least one free variable."(p.44) If this is true we can also state that the contrapositive is true. The equation $Ax=b$ has no free variables if and only if the homogeneous equation $Ax=0$ has no non trivial solutions. Since we have an if and only if statement then the reverse is true as well. If the homogeneous equation $Ax=0$ has no non trivial solutions then the equation $Ax=b$ has no free variables. Since $Ax=b$ has no free variables its solution must be unique.
\vspace{1in}







%%%third problem
\noindent\textbf{Exercise 1.5.31: }Suppose $A$ is a 3X􏰍2 matrix with two pivot positions.

\begin{enumerate}[(a)]
\item Does the equation $Ax=0$ have a nontrivial solution. \\
\noindent \textbf{Answer: }It does not have a nontrivial solution, because there is a pivot in each column which means there are no free variables.
\vspace{1in}

\item Does the equation $Ax=b$ have at least one possible solution for every possible $b$. \\
\noindent \textbf{Answer: }The equation $Ax=b$ will only have a solution when 


\begin{equation}
b=\begin{bmatrix}
b_1\\
b_2\\
0
\end{bmatrix}
\end{equation}
If $b_3\neq0$ then $Ax=b$ has no solution.



\vspace{1in}


\end{enumerate}
\vspace{1in}










\end{document}



















